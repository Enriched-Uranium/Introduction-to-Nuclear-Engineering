\section{Nuclear Reactor Theory}

\subsection{One-Group Reactor Equation}

\begin{equation}
    D\nabla^2 \phi - \varSigma_a \phi + S = \frac{1}{v}\pdv{\phi}{t}
\end{equation}

Since, $S = k_{\infty} \varSigma_a \phi$,  use multiplication factor $k_{\rm eff}$ to balance the equation, time-independent neutron diffusion equation is

\begin{equation}
    D\nabla^2 \phi - \varSigma_a \phi + \frac{k_{\infty}}{k_{\rm eff}} \varSigma_a \phi = 0
    \label{eq3.2}
\end{equation}

The form is Helmholtz equation, 

\begin{equation}
    \nabla^2 \phi + B^2 \phi = 0
    \label{eq3.3}
\end{equation}

the solution depends on the shape of the reactor.

When the reactor is critical, $k_{\rm eff} = 1$, by equation \ref{eq3.2} and \ref{eq3.3}, we have {\itshape Material Buckling}

\begin{equation}
    B^2 = \frac{k_{\infty} - 1}{L^2} = B_m^2
\end{equation}

At any time, {\itshape Geometric Buckling} equals to Material Buckling,

\begin{equation}
    B_g^2 = B_m^2
\end{equation}

must be satisfied for the reactor to be critical.

Alternative expression of critical equation is
\begin{equation}
    \frac{k_{\infty}}{1 + L^2 B_g^2} = 1
\end{equation}

Modified: $L^2 \longrightarrow M^2 = L^2 + \tau$.

\subsection{Thermal Reactors}

\begin{definition}[Thermal Reactor]
    A nuclear reactor with a large fission rate by thermal neutrons.
\end{definition}

\subsubsection*{Six quantities}

\begin{enumerate}
    \item The fast fission factor
    \begin{equation}
        \varepsilon = \frac{\text{number of fast neutrons}}{\text{number of fast neutrons produced by thermal fissions}}
    \end{equation}
    \item The fast non-leakage probability
    
    $P_s$ is the fraction of the fast neutrons that do not leak from the reactor.
    \item The resonance escape probability
    
    $p$ is the fraction of neutrons that pass through the resonance region without being absorbed.
    \item The thermal non-leakage probability
    
    $P_t$ is fraction of the thermal neutrons that do not leak from the reactor.
    \item The thermal utilization factor
    \begin{equation}
        f = \frac{\text{number of neutrons absorbed in the fuel}}{\text{total number of thermal neutrons absorbed}}
    \end{equation}
    \item The number of effective fission neutrons
    \begin{equation}
        \eta = \frac{\text{average number of fission neutrons released}}{\text{number of neutrons absorbed in the fuel}}
    \end{equation}
\end{enumerate}

\begin{equation}
    k_{\rm eff} = \varepsilon p f \eta P_s P_t = k_{\infty} P_L
\end{equation}

\subsection{Reflected Reactors}

Reflector
\begin{itemize}
    \item Flatten the radial neutrons flux density distribution
    \item Improve core edge fuel efficiency
    \item Reduce neutrons leakage and critical core radius
    \item Usually use the same material as the moderator
\end{itemize}